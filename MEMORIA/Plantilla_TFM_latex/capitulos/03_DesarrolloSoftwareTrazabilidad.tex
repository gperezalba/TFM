\chapter{Desarrollo de un software de análisis de trazabilidad de tiempo}

\section{Introducción}

Como se describe en los objetivos del trabajo, una de las primeras tareas es conseguir un mecanismo con el cuál comprobar la sincronización de dos relojes. Sin este mecanismo el trabajo carece de sentido ya que es vital conocer la precisión de la sincronización entre relojes. Denominado trazabilidad de tiempo, este mecanismo es a menudo usado por diversos grupos de trabajo relacionados con el tiempo y el posicionamiento para algunos propósitos específicos.\newline

Pese a ser conocido para estos grupos, es necesario disponer de los dispositivos, software y bibliotecas requeridos para su correcto funcionamiento. Todo este procedimiento se recoge en el presente apartado.

\section{Generación de ficheros RINEX}
Entre otros, el método más completo para la generación de ficheros RINEX consiste en utilizar un dispositivo de posicionamiento GPS para la recepción de los datos. Este tipo de dispositivos están preparados para recibir las señales emitidas por las diversas constelaciones de satélites de los sistemas globales de navegación por satélite. La información recibida por dichos dispositivos se puede emplear para visualizar diversos parámetros como la localización o almacenar dicha información. Cada dispositivo grabará la información en un determinado formato propio y será necesario el uso de herramientas específicas para transformar dichos ficheros a un formato estandarizado si se necesita.

\subsection{Dispositivo U-blox}
XXXXXXXXXXCambiar al ublox de JoseLuisXXXXXXXXXXXX
El dispositivo U-blox EVK-M8F es un dispositivo de posicionamiento que cumple con el propósito anteriormente descrito. \newline

\begin{figure}
	\centering
	\includegraphics[width=1\textwidth]{imagenes/ublox.png}
	\caption{\label{fig1}U-blox EVK-M8F \cite{ublox}.}
\end{figure}

El dispositivo se conecta al ordenador mediante USB del que recibe la alimentación. A su vez se conecta una antena que debe situarse en algún espacio al aire libre en el que la recepción de la señal sea apropiada. La conexión por USB al ordenador proporciona al mismo tiempo los datos que se reciben. \newline

La empresa U-blox proporciona software para la correcta interacción con el dispositivo y los datos que este presta. U-center \cite{ucenter} es el software GNSS que soporta todos los dispositivos u-blox y permite visualizar e interactuar con los datos recibidos por este. Su interfaz gráfica permite visualizar los satélites de los que se recibe información, geolocalización, parámetros relacionados con la señal recibida, el tiempo, etc. La ventana de configuración permite indicar al dispositivo los mensajes que se quiere que este transmita. También se puede configurar la constelación de la que se quiere que el receptor reciba y envíe información. Los posibles datos a recibir se engloban en dos tipos, (1) NMEA (National Marine Electronics Association), estándar para la interacción con los dispositivos marinos electrónicos y (2) UBX, protocolo binario para los mensajes de posicionamiento. \newline

\begin{figure}
	\centering
	\includegraphics[width=1\textwidth]{imagenes/gnssconfig.jpg}
	\caption{\label{fig1}Configuración receptor U-blox desde u-center.}
\end{figure}

La información relevante para este trabajo se encuentra en los mensajes UBX-RXM-RAWX (Multi GNSS RAW Measurement Data). Estos mensajes lanzarán la información relacionada con el posicionamiento GNSS. Se activan por tanto estos mensajes y se procede al grabado de un fichero que contenga estos mensajes. Entre las herramientas se encuentra "Record" que permite generar un fichero en formato ublox que almacena estos mensajes durante tanto tiempo como se desee. Notar que los mensajes se encuentran en formato raw en este fichero. \newline

\begin{figure}
	\centering
	\includegraphics[width=1\textwidth]{imagenes/messagesview.jpg}
	\caption{\label{fig1}Ventana u-center para visualizar y habilitar mensajes.}
\end{figure}

\subsection{RTKLIB}
La librería de aplicaciones para el posicionamiento GNSS contiene una serie de aplicaciones para interactuar con todo lo relacionado con este sistema. \newline

En su repositorio de código \cite{reportklib} se encuentran los ejecutables de las distintas aplicaciones que ofrece. La aplicación CONVBIN permite convertir ficheros en diversos formatos a ficheros RINEX. Los ficheros generados en el apartado anterior sirven de entrada para esta aplicación. A la aplicación es necesario indicarle tanto el formato de entrada de los ficheros (ubx) como las salidas que se quiere que proporcione (ficheros de observaciones y de navegación).

La llamada a la aplicación se realiza por la terminal con el siguiente formato:

\begin{lstlisting}
>> ./convbin.exe -r ubx ./rawData.ubx -o file.obs -n file.nav -g file.gnav
\end{lstlisting}

Una ejecución exitosa de la aplicación nos genera los ficheros necesarios para las próximas tareas.

\subsection{Otros ficheros RINEX disponibles}
Existen muchos grupos que proporcionan de manera abierta todos los ficheros RINEX que graban. Uno de estos grupos es el Royal Observatory of Belgium que almacena todos sus ficheros RINEX en un servidor FTP \cite{ftprob} para que se puedan descargar de manera libre. \newline

Se tiene en cuenta esta posibilidad ya que como se verá en el apartado de "Generación de ficheros CGGTTS", no todos los ficheros RINEX son válidos para generar un CGGTTS.\newline


\section{Generación de ficheros CGGTTS}
El fichero CGGTTS contiene la información relacionada con la trazabilidad de tiempo del espacio temporal del que se tiene el fichero RINEX de un determinado dispositivo. \newline

El software encargado de generar ficheros CGGTTS a partir de ficheros RINEX se llama R2CGGTTS. Este software es desarrollado, mantenido y distribuido por el grupo Royal Observatory of Belgium. Esta herramienta está disponible en el FTP del BIPM \cite{ftpr2cggtts}. \newline

Al descargar la herramienta se observa que existen varias versiones de la misma. Durante los primeros meses de trabajo, la última versión disponible era la 7, así que al principio se trabaja con estas. En la documentación disponible para cada versión se observan las novedades que ha ido incluyendo cada una. Los aspectos más relevantes a tener en cuenta son la versión del fichero RINEX y las constelaciones GNSS que son soportadas. \newline

Para comenzar a trabajar con la herramienta el primer paso consiste en compilar el código con un compilador Fortran 77. 

\begin{lstlisting}
>> gfortran R2CGGTTS_V7.f -o R2CGGTTS_executable
\end{lstlisting}

Una vez compilado y generado el ejecutable, ya está listo para su ejecución. Antes de ejecutarlo hay que tener en cuenta que (1) los ficheros RINEX se deben colocar en el mismo directorio que el ejecutable y (2) se deben rellenar dos ficheros especiales (paramCGGTTS.dat e inputFile.dat). Estos dos ficheros contienen los parámetros relacionados con el receptor y los nombres de los ficheros RINEX de entrada. Ahora ya se puede ejecutar el código. Desde una terminal habrá que desplazarse al directorio correspondiente y lanzar el ejecutable.

\begin{lstlisting}
>> ./R2CGGTTS_executable
\end{lstlisting}

La salida esperada tras una ejecución exitosa es la creación en el mismo directorio de los ficheros CGGTTS.

\subsection{Problemas con la herramienta}
Pese a que fueron varios los problemas acontecidos durante el trabajo, hay dos que son los más representativos y que merece la pena mencionar.

\subsubsection{Versión del fichero RINEX}
Cada versión nueva del software R2CGGTTS ha ido incluyendo novedades a lo largo del tiempo. Una de ellas fue el cambio de versión (de la versión 2.x a la 3.x) de los ficheros RINEX entre las versiones 5 y 7 de la herramienta. El procedimiento seguido para generar un fichero RINEX (descrito en el apartado "Generación de ficheros RINEX") genera un fichero RINEX de versión 2.11. Esta versión es soportada solamente por la versión 5 del software R2CGGTTS. En paralelo a esto se prueba la versión 7 de R2CGGTTS con ficheros RINEX versión 3 disponibles online para su descarga.  \newline

En este tiempo se hace uso de herramientas que permiten la conversión de versiones de los ficheros RINEX \cite{gnssconverter}. Algunas de estas herramientas también permiten la conversión de ficheros raw en formato ublox a la versión deseada de fichero RINEX. Sin embargo, de ninguna herramienta se obtiene la salida esperada. \newline

Durante este tiempo de trabajo aparece la versión 8 del software R2CGGTTS. Entre las novedades que incluye, la principal consiste en permitir todas las versiones de ficheros RINEX. Cuando se subsana este problema se está en perfecta disposición para crear los ficheros CGGTTS.

\section{Trazabilidad de tiempo con ficheros CGGTTS}
Una vez se ha generado un fichero CGGTTS propio se necesita el correspondiente fichero CGGTTS de otro receptor GPS (presumiblemente de un laboratorio) con el que se quiera comparar. Se elige el Real Instituto Observatorio de la Armada (ROA) \cite{roa} como laboratorio con el que realizar la trazabilidad de tiempo. En el servidor FTP del ROB \cite{ftpro} se colocan diariamente los ficheros CGGTTS de los principales laboratorios de varios países (entre ellos España y el ROA). 

\subsection{Software de procesado de ficheros CGGTTS}
Con 2 ficheros CGGTTS del mismo día procedentes de dos receptores GPS distintos se tiene todo lo necesario para comprobar las desviaciones que tuvo un reloj respecto al otro dicho día. \newline

\begin{figure}
	\centering
	\includegraphics[width=1\textwidth]{imagenes/cggtts.PNG}
	\caption{\label{fig1}Ejemplo de fichero CGGTTS \cite{timetransfer}.}
\end{figure}

Como se observa en la Figura 3.4, entre la información que contiene el fichero CGGTTS la columna REFGPS será vital para conseguir la trazabilidad temporal. La columna REFGPS contiene la diferencia de hora del receptor GPS respecto a una referencia temporal que será la misma para los dos ficheros CGGTTS. Al contener la misma referencia temporal, se puede computar la desviación entre relojes como:

\begin{equation}
\Delta t = REFGPS(local) - REFGPS(ROA) 
\end{equation}

\begin{equation}
\Delta t = (UTC(local) - REF) - (UTC(ROA) - REF)
\end{equation}

\begin{equation}
\Delta t = UTC(local) - UTC(ROA)
\end{equation}

La técnica que permite esta comparación se denomina \textit{"Satelites in common view"}, por lo que se tendrán que computar las diferencias solamente entre los satélites que se encuentren en ambos ficheros CGGTTS. En la columna denominada PRN de la Figura 3.4 se observan el identificador del satélite que proporciona la medida de dicha fila. Para una misma franja temporal (indicada por la columna STTIME) existen varios satélites que proporcionan medidas. Por lo tanto, el software debe computar las diferencias entre columnas REFGPS cuando exista una medida de un satélite en una misma franja temporal en los dos ficheros CGGTTS y obtener un promedio para cada franja temporal. Surgen algunas dudas sobre cómo sería la manera más óptima de realizar el promedio y de si sería apropiado utilizar alguna ponderación para las medidas en función de algún otro parámetro. Se decide entonces plantear estas cuestiones al ROA y en su respuesta proporcionan ciertos scripts que ellos utilizan para computar la trazabilidad temporal entre dos ficheros CGGTTS. \newline

Los scripts proporcionados por el ROA (Anexo XXXXXXXXXXXXXXXX) realizan todos los pasos descritos en esta sección. Dichos pasos se pueden resumir en los siguientes puntos:

\begin{itemize}
	\item Lectura de ambos ficheros CGGTTS.
	\item Satélites en \textit{common view} para cada franja temporal.
	\item Computar la diferencia para cada instante y cada satélite.
	\item Se realiza un promedio ponderado para todas las medidas disponibles para una misma franja horaria. La ponderación se basa en la columna ELV que indica la elevación a la que se encuentra el satélite en cuestión, factor determinante para la calidad de la medida.
	\item Dibujar una figura con las diferencias obtenidas.
\end{itemize}

En la figura 3.5 se observa la figura que genera dicho script. Las diferencias obtenidas se presentan de varias formas: \textit{raw}, la media de cada franja temporal, con un filtro 3 sigma y con un filtrado Vondrak que reduce el ruido. La interpretación de la figura indica que la desviación del reloj del receptor GPS colocado en la UGR con respecto al receptor GPS de ROA se encuentra entre -40 ns y +30 ns. Se obtiene así la precisión de la sincronización entre dos relojes haciendo uso de la técnica satélites en \textit{common view}.

\begin{figure} 
	\centering
	\includegraphics[width=1\textwidth]{imagenes/resultado_cv2.PNG}
	\caption{\label{fig1}Trazabilidad de tiempo entre UGR y ROA.}
\end{figure}


